\documentclass{article}
\author{Max Springenberg}
\title{PhB2 UB0}
\setcounter{section}{0}
\linespread{1.5}
\usepackage{gensymb}
%\usepackage{amsmath}
%\usepackage{amssymb}
%\usepackage{stmaryrd}

\begin{document}
\maketitle
\newpage

\subsection{Die Bieraufgabe}
Der Inhalt einer Flasche Bier (0,5 l) hat einen Brennwert (vom menschlichen Körper verwertbare,
chemische Energie) von 250 kcal (kcal = Kilokalorien). Die spezifische Wärmekapazität und die Dichte
von Bier entsprechen in etwa der des Wassers. Nehmen Sie daher für alle Aufgabenteile als spezifische
Wärmekapazität den Wert c Wasser = 4,186 kJ/(kg · K) an. Die Dichte von Wasser beträgt ρ =1 kg/l\\
\\
\subsubsection\
$
c_{Wasser} = 4,186 \frac{kJ}{kg*K}, \rho = 1 \frac{kg}{l}\\
\\
\Delta Q 
    = c_{Wasser} * m * \Delta T\\
    = (4,186 \frac{kJ}{kg*K} * 10^3) * 10^{-3}kg * 1K\\
    = 4,186 kJ\\
\\
W = 250*10^3*\Delta Q = 1,05MJ\\
$
\subsubsection\
Bedarf in MJ fuer 24h\\
$
p = 100W = 100 J/s\\
T = 24*60*60s = 86400s\\
W_{24h} = 8,64 MJ\\
\\
N = \frac{8,64 MJ}{1,05MJ} = 8,25
$
\subsubsection\
$
\Delta T = 33 \degree\\
\\
\Delta Q 
    = c * m * \Delta T\\
    = (4,186 \frac{kJ}{kg*K} * 10^3) * 0,5kg * 33K\\
    = 69,01kJ\\
    = 16,5 kcal ???\\
\\
\frac{16,5}{250}kcal = 6,6\%\\
$
\end{document}
