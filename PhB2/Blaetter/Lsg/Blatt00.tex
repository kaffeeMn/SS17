\documentclass{article}
\author{Max Springenberg}
\title{PhB2 UB0}
\setcounter{section}{0}
\linespread{1.5}
\usepackage{gensymb}
%\usepackage{amsmath}
%\usepackage{amssymb}
%\usepackage{stmaryrd}

\begin{document}
\maketitle
\newpage

\subsection{Die Bieraufgabe}
Der Inhalt einer Flasche Bier (0,5 l) hat einen Brennwert (vom menschlichen Körper verwertbare,
chemische Energie) von 250 kcal (kcal = Kilokalorien). Die spezifische Wärmekapazität und die Dichte
von Bier entsprechen in etwa der des Wassers. Nehmen Sie daher für alle Aufgabenteile als spezifische
Wärmekapazität den Wert c Wasser = 4,186 kJ/(kg · K) an. Die Dichte von Wasser beträgt ρ =1 kg/l\\
\subsubsection\
$
c_{Wasser} = 4,186 \frac{kJ}{kg*K}, \rho = 1 \frac{kg}{l}\\
\\
\Delta Q 
    = c_{Wasser} * m * \Delta T\\
    = (4,186 \frac{kJ}{kg*K} * 10^3) * 10^{-3}kg * 1K\\
    = 4,186 kJ\\
\\
W = 250*10^3*\Delta Q = 1,05MJ\\
$
\subsubsection\
Bedarf in MJ fuer 24h\\
$
p = 100W = 100 J/s\\
T = 24*60*60s = 86400s\\
W_{24h} = 8,64 MJ\\
\\
N = \frac{8,64 MJ}{1,05MJ} = 8,25
$
\subsubsection\
$
\Delta T = 33 \degree\\
\\
\Delta Q 
    = c * m * \Delta T\\
    = (4,186 \frac{kJ}{kg*K} * 10^3) * 0,5kg * 33K\\
    = 69,01kJ\\
    = 16,5 kcal ???\\
\\
\frac{16,5}{250}kcal = 6,6\%\\
$
\subsection{Erhitzen und Verdampfen}
Wie viel Wasser verdampft, wenn Sie 6 kg gluhende Stahlschrauben mit einer Temperatur von 1200 ◦ C
in 3 kg Wasser mit einer Temperatur von 20 ◦ C tauchen? Die spezifische Warmekapazität von Stahl be-
trage c Stahl =0,50 kJ/(kg·K) und die Verdampfungswärme des Wassers sei durch μ Wasser = 2257 kJ/kg
gegeben.\\
\\
\\
$
c_{Sahl} = 0,5 kJ/kg*K, \mu_H = 2257 kJ/kg\\
\Delta Q = c*m*\Delta T\\
    = 0,5 kJ/kg*K * 6kg * 2257 kJ/kg\\
    = 3540 kJ\\
    = 3,54 MJ\\
\\
m_{Dampf} = \frac{3,54MJ}{2,257MJ/kg} = 1,57 kg\\
$
\subsection{Ideales Gas und kinetische Energie}
In einem Ultrahoch-Hochvakuum, in dem nur molekularer Wasserstoff als stark verdünntes Gas, 
herrsche ein Druck von 1, 3 · 10 −11 mbar und eine Temperatur von 200 ◦ C.\\
\\
$
p = 1,3  *10^{-11} mbar = a,3 * 10^{-9} Pa\\
T = 200 \degree C = 473,15K\\
$
\subsubsection\
$
V = 1cm^3 = 1*10^{-6}m^3\\
\\
p*V = N*K_B*T\\
N   = \frac{p*V}{K_B*T}\\
    %= \frac{}
    = 2 * 10^5\\
$
\subsubsection\
$
U = 1/2 m * <v_i^2> = f/2 * K_B * T\\
\\
<v_i^2> 
    = f/m_i * K_B * T\\
    = \frac{5 * 1,38 * 10^{-23} J/k * 473K}{2*1,67 * 10^{-27}kg}\\
    = 9,76 * 10^6 m^2/s^2\\
\Rightarrow <v_i> = 3,124 m/s\\
$
\end{document}
