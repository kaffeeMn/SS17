\documentclass{article}
\author{Max Springenberg}
\title{DAP2-UB 27.4.17}
\usepackage{amsmath}
\usepackage{amssymb}
\usepackage{stmaryrd}
% \Theta \Omega \omega
\begin{document}
\maketitle
email Tutor:\\
anja.rey@tu-dortmund.de\\
\\
Abgabe Dienstag 12:00\\
keine Bloecker, insgesamt 50\%\\
\subsection{Definitionen}
$f,g: \mathbb{N} \rightarrow \mathbb{N}\\$
waechst hoechstens genauso schnell\\
$
f(n) \in O(g(n))\\
\exists c>0.\exists n_0.
    \forall n \geq n_0.f(n) \leq c*g(n)\\
$
\\
Die Untere Schranke\\
f waechst mindestens so schnell wie g\\
$
f(n) \in \Omega (g(n))\\
\exists c>0.\exists n_0.
    \forall n \geq n_0.f(n) \geq c*g(n)\\
$
wachst genauso schnell\\
$
f(n) \in \Theta (g(n))\\
f(n) \in O(n) \land f(n) \in \Omega (n)\\
$
echt weniger schnell\\
$
f(n) \in o(g(n))
\forall c>0.\exists n_0.
    \forall n \geq n_0.f(n) < c*g(n)\\
$
echt schneller\\
$
f(n) \in \omega (g(n))
\forall c>0.\exists n_0.
    \forall n \geq n_0.f(n) > geq c*g(n)\\
$
Aequivalenzen:\\
$
f(n) \in O(g(n)) \leftrightarrow g(n) \in \Omega(f(n))\\
f(n) \in o(g(n)) \leftrightarrow g(n) \in \omega(f(n))\\
$
\newpage

\subsection{Erste Schritte der O-Notation}
\subsubsection\
Beh:\\
$g(n) = n$\\
Bew:\\
$
n_0 \geq 202, c =2017\\
\Rightarrow f(n) \in O(n) \land f(n) \in \Omega(n)\\
$
\subsubsection\
Beh:\\
$h(n) = log(n)$\\
Bew:\\
$
log(n) \in o(n)\\
2017n \in \omega(log(n))\\
$
\subsection{Laufzeitanalyse:Primfaktoren}
\subsubsection\
\end{document}
