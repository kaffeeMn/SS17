\documentclass{article}
\author{Max Springenberg, 177792}
\title{
    DAP2 UB8\\
    Anja Rey, Gr.23 , Briefkasten 22
}
\usepackage{amsmath}
\usepackage{amssymb}
\usepackage{stmaryrd}
\setcounter{section}{8}
% \Theta \Omega \omega
\newcommand{\tab}{\null \qquad}
\newcommand{\lA}{$\leftarrow$}
\begin{document}
\maketitle
\subsection{Dynamische Programmierung}
gegeben:\\
n Uebungsblaetter\\
$1 \leq i\leq n$\\
Liste mit bepunktungen $A \in \mathbb{N}^{n}_{\geq 0}$\\
\subsubsection\
B mit $[b_0 ... b_n]$\\
Existieren keine Uebungsblaetter, so koennen auch keine Punkte vergeben werden, daher $B[0] = 0$\\
Existiert nur ein Uebungsblatt, so ist dessen Punktzahl auch die maximale Punktzahl, daher $B[1] = A[1]$\\
Ansonsten muss zwischen der maximalen Punktzahl bis zum vorherigen Uebungsblattes und der maximalen
Punktzahl bis zum Uebungsblatt davor plus der Punktzahl fuer das aktuelle Uebungsblatt selbst abgewogen werden.\\
$
B[i] = \begin{cases}
    0   & (i = 0)\\
    A[i]& (i = 1)\\
    max\{
        B[i-1], A[i] + B[i-2]
    \}  & sonst\\
\end{cases}
$
\subsubsection\
Uebungsblaetter(A):
1\tab n \lA length(A)\\
2\tab B \lA new Array[0..n]\\
3\tab B[0] \lA 0\\
4\tab B[1] \lA A[1]\\
5\tab for i \lA 2 to n do\\
6\tab \tab B[i] \lA max{B[i-1], A[i] + B[i-2]}\\
7\tab return B[n]\\
\subsubsection\
1-4: Konstante Laufzeit $\Theta(4)$\\
5-6: $\Theta(1+n*1)$\\
7: Konstante Laufzeit $\Theta(1)$\\
Insgesamt:\\
$
\Theta(4+1+n*1+1) = \Theta(n+5) \in O(n)\\
$
\subsubsection\
Aussage:\\
B[i] sei die Maximale Punktzahl fuer Alle Blaetter bis A[i].\\
Induktion ueber i.\\
\\
I.A.\\
i = 1:\\
Da es keine weiteren Blaetter gibt ist B[i] = A[i], damit korrekt.\\
\\
I.V.\\
Die Aussage gelte fuer $i' \in mathbb{N}$  mit $0 < i'< i \leq n$ beliebig, aber fest.\\
\\
I.S.\\
fuer i > i':
Annahme: m > B[i] sei die maximale Punktzahl fuer alle Blaetter bis A[i]\\
Das Maximum ist entweder:\\
(i) die maximale Punktzahl bis zum vorherigen Blatt A[i-1], oder\\
(ii) die Punktzahl vom aktuellem Blatt A[i] plus der maximalen Punktzahl bis zwei Blaetter zuvor A[i-2]\\
\\
(i) wird mit i > i-1 nach der I.V. durch B[i-1] ermittelt\\
(ii) das maximum bis A[i-2] wird nach der I.V. ebenfalls durch B[i-2] ermittelt.\\
demnach gilt:\\
max\{ maximale Punktzahl(A[i-1]), A[i] + maximale Punktzahl(A[i-2]) \}
$
= max\{B[i-1], A[i] + B[i-2]\} = B[i]\\
$
Demnach kann m nicht groesser B[i] sein und die Aussage ist bestaetigt.\\

\newpage
\subsection{Dynamische Programmierung}
gegeben:\\
n Bruecken\\
$
i, n \in \mathbb{N}, 1 \leq i \leq n\\
A, B \in \mathbb{N}^{n-1}_{\geq 0}, C \in \mathbb{N}^{n}_{\geq 0}\\
C[1] = C[n] = 0\\
$
\subsubsection\
$\forall i \leq 0$ gilt: es wurden noch keine Brezeln gesammelt oder abgegeben.\\
$\forall i > 1$ gilt: es koennen bereits Brezeln gesammelt und wieder abgegben wurden sein.
Damit muss der erwerbt von $A[i], B[i]$ mit dem jeweiligen Fall einer Abgabe in $C[i+1]$ abgewogen werden
und der maximale Wert genommen werden.\\
\\
Fuer p(i), dass die maximale Brezeln-Anzahl in Buda angibt und q(i), dass analog fuer Pest funktioniert
ergeben sich somit die Rekursionsgleichungen:\\
$
p(i) = \begin{cases}
    0   & (i \leq 1)\\
    max\{
        A[i] + p(i-1),
        B[i] + q(i-1) - C[i+1]
    \}  & sonst\\
\end{cases}\\
q(i) = \begin{cases}
    0   & (i \leq 1)\\
    max\{
        A[i] + p(i-1) - C[i+1],
        B[i] + q(i-1)
    \}  & sonst\\
\end{cases}\\
$
\subsubsection\
Bruecken(A, B, C):\\
1\tab n \lA min{length(a), length(B)}\\
2\tab Buda \lA new Array[1..n]\\
3\tab Pest \lA new Array[1..n]\\
4\tab Pest[0] \lA 0\\
5\tab Buda[0] \lA 0\\
6\tab for i \lA 2 to n do\\
7\tab \tab Buda[i] \lA $max\{\\
\tab \tab \tab A[i] + Buda[i-1],\\
\tab \tab \tab B[i] + Pest[i-1] - C[i+1]\\
\tab \tab \}$\\
8\tab \tab Pest[i] \lA $max\{\\
\tab \tab \tab A[i] + Buda[i-1] - C[i+1],\\
\tab \tab \tab B[i] + Pest[i-1]\\
\tab \tab \}$\\
9\tab return Buda[n]\\
\subsubsection\
1-5: Konstante Laufzeit $\Theta(5)$\\
6-8: $\Theta(1+n*2)$\\
9: Konstante Laufzeit $\Theta(1)$\\
\\
Insgesamt damit:\\
$
\Theta(5+1+n*2+1) = \Theta(2*n + 7) \in O(n)\\
$
\subsubsection\
Aussage:\\
p(i) berechnet die maximale Anzahl an Brezeln in Buda und q(i) berechnet die maximale Anzahl an Brezeln in Pest.\\
Induktion ueber i.\\
\\
I.A.\\
i = 1:\\
Zu Beginn gibt es noch keine Brezeln und es wurden auch noch keine abgenommen, eine Brueckenueberfuehrung ist kostenlos,
daher ist das Ergebnis fuer alle Faelle 0 = q(1) = p(1).\\
\\
I.V.\\
Die Aussage gelte fuer $i' \in mathbb{N}$  mit $0 < i'< i \leq n$ beliebig, aber fest.\\
\\
I.S.\\
fuer i = i' + 1:\\
1.\\
p(i) muss fuer i > 1 ein Maximum ermitteln:\\
es kann entweder der Weg vom letzten Maximum p(i-1) ueber
A[i] mit:\\
(p(i-1) + A[i])\\
\\
oder der Weg vom letzen Maximum des anderen Ufers q(i-1) ueber
B[i] unter Subtraktion des Zolles der Bruecke C[i+1] mit:\\
(q(i-1) + B[i] - C[i+1])\\
\\
zum Maximum führen.\\
Dabei sind q(i-1) und p(i-1) nach I.V. korrekt.\\
demnach liefert:\\
max\{(p(i-1) + A[i]), (q(i-1) + B[i] - C[i+1]\}\\
offensichtlich das Maximum fuer p(i)\\
\\
Analog fuer q(i):\\
q(i) muss fuer i > 1 ein Maximum ermitteln:\\
es kann entweder der Weg vom letzten Maximum q(i-1) ueber
B[i] mit:\\
(q(i-1) + B[i])\\
\\
oder der Weg vom letzen Maximum des anderen Ufers p(i-1) ueber
A[i] unter Subtraktion des Zolles der Bruecke C[i+1] mit:\\
(p(i-1) + A[i] - C[i+1])\\
\\
zum Maximum führen.\\
Dabei sind q(i-1) und p(i-1) nach I.V. korrekt.\\
demnach liefert:\\
max\{(q(i-1) + B[i]), (p(i-1) + A[i] - C[i+1]\}\\
offensichtlich das Maximum fuer p(i)\\
\\
Damit ist die Aussage bestätigt.
%Annahme: \\
%(i) $m_q$ > q(i) sei maximaler Brezeln-Wert fuer Pest und 
%(ii) $m_p$ > p(i) fuer Buda.\\
\end{document}
